\documentclass[12pt]{article}

\usepackage[T1]{fontenc}
\usepackage[polish]{babel}
\usepackage[utf8]{inputenc}
\usepackage{lmodern}
\selectlanguage{polish}
\usepackage{amsfonts}
\usepackage{indentfirst}

\begin{document}

\title{
\textbf{Pracownia Programowania Obiektowego\\}
\textbf{Lista nr 6 - 08.04.2018\\}
\textbf{Język programowania:} Java\\
\textbf{OS:} Windows 10 Enterprise 64-bit\\
\textbf{IDE i kompilator:} JetBrains IntelliJ IDEA\\
\textbf{Project SDK:} Java 10\\
Opis implementacji zadań 1 oraz 2
}

\author {Mateusz Zając, 298654}

\date {08.04.2018} 

\maketitle

\newpage

\section{Zadanie 1}

Zadanie polega na zaimplementowaniu w javie dowolnej kolekcji już wcześniej napisanej na poprzedniej liście.
Moim wyborem była Lista, która miała służyć zarówno jako stos jak i kolejka. Struktura ta ma odnośniki zarówno do pierwszego jak i ostatniego elementu, a także możliwość przejścia do przodu jak i wstecz kolekcji.
Dodatkowo, struktura implementuje interfejs \textbf{\textit{Serializable}}, aby kolekcję można było zapisać i odczytać z dysku. \\

Struktura nie ma w sobie większych zmian względem wcześniej zaimplementowanej struktury z listy 3. 

\newpage

\section{Zadanie 2}
	
Zadanie jest podobne do poprzedniego, jednak struktura ma implementować interfejs \textbf{\textit{Collection}}.
Nie zdążyłem jednak zaimplementować wszystkich metod poprawnie. Przez brak implementacji metody \textbf{\textit{iterator}} część napisanego przeze mnie programu nie działa poprawnie.

Opis metod zaimplementowanych w programie dostępny jest w samych plikach programu.
	 
\end{document}