\documentclass[12pt]{article}

\usepackage[T1]{fontenc}
\usepackage[polish]{babel}
\usepackage[utf8]{inputenc}
\usepackage{lmodern}
\selectlanguage{polish}
\usepackage{amsfonts}
\usepackage{indentfirst}

\begin{document}

\title{
\textbf{Pracownia Programowania Obiektowego\\}
\textbf{Lista nr 7 - 15.04.2018\\}
\textbf{Język programowania:} Java\\
\textbf{OS:} Windows 10 Enterprise 64-bit\\
\textbf{IDE i kompilator:} IntelliJ IDEA\\
Opis implementacji zadań z listy 7
}

\author {Mateusz Zając, 298654}

\date {15.04.2018} 

\maketitle

\newpage

\section{Podstawy implementacji klas}

Przed przystąpieniem do realizacji zadań z listy musimy utworzyć odpowiednią hierarchię klas.
Naszą klasą abstrakcyjną, po której dziedziczą obie podklasy jest \textbf{\textit{Figura}}.
Wspólną częścią podklas \textbf{\textit{Okrag}} oraz \textbf{\textit{Trojkat}} są pola:
\begin{itemize}
\item \textbf{\textit{String name}} - nazwa figury
\item \textit{\textbf{String colour}} - kolor figury
\end{itemize}
Pozostałe pola:
\begin{itemize}
\item \textbf{\textit{double radius}} (pole klasy \textbf{\textit{Okrag}}) - promień okręgu
\item \textbf{\textit{double base}} (pole klasy \textbf{\textit{Trojkat}}) - podstawa trójkąta
\item \textbf{\textit{double height}} (pole klasy \textbf{\textit{Trojkat}}) - wysokość trójkąta
\end{itemize}

Z tak utworzoną hierarchią możemy przystąpić do właściwej części listy zadań.

\section{Pierwsza część}

	Pierwsza część polega na implementacji interfejsu do edycji obiektów. Dzięki bibliotece \textbf{\textit{Swing}} mamy możliwość wyświetlenia na ekranie okna dialogowego z odpowiednimi polami tekstowymi i przyciskami, które są niezbędne do edycji pól obiektu. Wymagany z zadaniu interfejs jest realizowany przy pomocy metody \textbf{\textit{guiFig}} odpowiednio klas \textbf{\textit{Okrag}} i \textbf{\textit{Trojkat}}. Na początku metoda wywołuje \textbf{\textit{loadObject}}, opisaną w drugiej części zadania. Wczytujemy z pliku obiekt do edycji. Kolejne linijki metody mają na celu zdefiniowanie pól edycji, ich etykiet oraz przycisku potwierdzającego zapis wpisanych danych do pól obiektu. Do obiektu \textbf{\textit{but1}} (przycisku ,,zapisz'') dopisujemy ,,słuchacza''. Dzięki niemu kiedy tylko użytkownik naciśnie przycisk, zostanie wywołane parsowanie zawartości pól i ich zapis (za pomocą metody \textbf{\textit{saveObject}}). Argument przekazywany do metody \textbf{\textit{guiFig}} jest używany do określenia jaki plik ma czytać program i do jakiego pliku po edycji ma zapisywać nowe wartości pól.
	
\section{Druga część}

	Druga część zadania to implementacja interfejsu \textbf{\textit{Serializable}} razem z metodami służącymi do zapisu oraz odczytu obiektów z pliku:
	\begin{itemize}
	\item \textbf{\textit{loadObject(String s)}} - Czyta plik o nazwie podanej z argumencie. Jeżeli podany plik nie istnieje, inicjowany jest nowy obiekt z domyślnymi wartościami pól. 
	\item \textbf{\textit{saveObject(String s)}} - Zapisuje obiekt do pliku o nazwie podanej w argumencie. 
	\end{itemize}

\section{Uruchomienie programu}

	W celu uruchomienia programu musimy poza wywołaniem pliku wykonalnego programu z linii poleceń podać także nazwę szukanego pliku z zapisanym obiektem (bez rozszerzenia) oraz nazwę klasy obiektu, który jest zawarty w pliku.
	
	 
\end{document}