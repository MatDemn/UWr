\documentclass[12pt]{article}

\usepackage[T1]{fontenc}
\usepackage[polish]{babel}
\usepackage[utf8]{inputenc}
\usepackage{lmodern}
\selectlanguage{polish}
\usepackage{amsfonts}
\usepackage{indentfirst}

\begin{document}

\title{
\textbf{Pracownia Programowania Obiektowego\\}
\textbf{Lista nr 5 - 01.04.2018\\}
\textbf{Język programowania:} Java\\
\textbf{OS:} Windows 10 Enterprise 64-bit\\
\textbf{IDE i kompilator:} JetBrains IntelliJ IDEA\\
\textbf{Project SDK:} Java 10\\
Opis implementacji zadań 1 oraz 2
}

\author {Mateusz Zając, 298654}

\date {01.04.2018} 

\maketitle

\newpage

\section{Zadanie 1}

Zadanie polega na zaimplementowaniu kolekcji przechowującej figury w kolejności rosnącej, według ich pól powierzchni. \\
\textit{(Hierarchia klas dostępna w załączonym schemacie)} \\

\textbf{\textit{Klasa Figura}} - Stanowi schemat tworzenia figur geometrycznych. Jest to klasa abstrakcyjna, ponieważ nie możemy jednoznacznie wyznaczyć pola nieznanej figury, nie znając jej właściwości i kształtu. Zawiera w sobie składowe różnych figur geometrycznych, jednak nie każda podklasa korzysta z wszystkich pól klasy nadrzędnej. 

\textit{Przykładowo: \textbf{\textit{Klasa Trojkat}} korzysta jedynie z pól: \textbf{\textunderscore bok1} oraz \textbf{\textunderscore wysokosc}, ponieważ tylko te parametry są potrzebne aby wyliczyć pole trójkąta.}\\

\textbf{\textit{Klasy: Kolo, Trojkąt, Prostokat, Trapez}} - Obiekty tych klas reprezentują powyższe kształty geometryczne. \\

\textbf{\textit{Klasa FigCollection}} - Służy za przechowywanie elementów Node, tworzenie listy elementów, dodawanie oraz usuwanie z listy. 

Z uwagi, iż kolekcja jest uporządkowana, dodawanie odbywa się na podstawie porządku już istniejącej listy. Algorytm nie wstawia nowego elementu dowolnie, ale porównuje pola poszczególnych elementów i zamieszcza nową figurę we właściwe miejsce.\\ 


\textbf{\textit{Klasa Node}} -
Obiekty przechowują dowolne figury geometryczne, które są określone w programie. W skład obiektu \textbf{\textit{Node}} wchodzi figura oraz odnośnik do kolejnego elementu \textbf{\textit{Node}}, aby zachować porządek i ciągłość struktury danych.
\newpage

\section{Zadanie 2}
	
Zadanie polega na utworzeniu odpowiednich klas, które reprezentować będą podstawowe wyrażenia arytmetyczne tj. +, -, *, / oraz ich wyliczanie.


\textbf{\textit{Klasa Wyrazenie}} -
Główny element programu. Klasa, która jest schematem dla wszystkich zdefiniowanych operacji arytmetycznych. Zawiera w sobie schematy metod dla podklas.\\

\textbf{\textit{Klasy: Add, Sub, Mul, Div}} - Opisują kolejno dodawanie, odejmowanie, mnożenie oraz dzielenie. Przysłaniają metody klasy nadrzędnej \textbf{\textit{Wyrazenie}}, definiując własne metody. Każda z operacji ma inny sposób wypisywania oraz obliczania swoich składowych (\textbf{\textit{\textunderscore leftside}} oraz \textbf{\textit{\textunderscore rightside}}). \\

\textbf{\textit{Klasa Div}} różni się od pozostałych pod względem konstrukcji metody \textbf{\textit{oblicz();}}. Jak wiemy, przy dzieleniu musimy sprawdzić czy mianownik nie jest równy liczbie zero. Dlatego też kiedy metoda napotka w argumencie \textbf{\textit{\textunderscore rightside}} liczbę zero, wyrzuca wyjątek w konsoli, nie zatrzymując przy tym działania programu.\\

\textbf{\textit{Klasa Stala}} - Opisuje stałe, czyli po prostu liczby. Jest to najmniej złożona klasa. Metoda wyliczająca wartość zwraca po prostu jedyne pole klasy.\\

\textbf{\textit{Klasa Zmienna}} - Opisuje zmienne. Do wyliczenia wartości wyszukuje w tablicy haszującej wartość zmiennej i zwraca ją. Jeśli zmienna nie występuje w tablicy, zostaje zwrócony wyjątek.
	 
\end{document}